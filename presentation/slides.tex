\documentclass{beamer}

\usetheme{Darmstadt}
\usecolortheme{rose}

\usepackage[utf8]{inputenc}
\usepackage[english]{babel}
\usepackage{hyperref}
\usepackage[autostyle]{csquotes}

\usepackage {tikz}
\usetikzlibrary {positioning}

\title{The Adaptive Priority Queue with Elimination and Combining}
\subtitle{Irina Calciu, Hammurabi Mendes, and Maurice Herlihy}
\author{Stefan Haider\\ \scriptsize\href{mailto:stefan.haider@student.tuwien.ac.at}{stefan.haider@student.tuwien.ac.at}}
\institute{184.755 Seminar aus Programmiersprachen}
\date{\today}

\begin{document}
		
	\begin{frame}[plain]
		\titlepage
	\end{frame}
	
	\section{Graph theory}
	\subsection{What is a graph?}
	\begin{frame}
		\frametitle{What is a graph?}
		\framesubtitle{Just a short recap...}
		\begin{block}{Definition}
			\begin{itemize}
				\item A graph is an ordered pair $G=(V,E)$
				\item $V$ is a set of vertices/nodes
				\item $E$ is a set of (un)directed (weighted) edges/arcs
				\item each edge is a two element subset of $V$
			\end{itemize}
		\end{block}
	\end{frame}
	
	\subsection{Hamiltonian path/cycle/graph}
	\begin{frame}
		\frametitle{Hamiltonian path/cycle/graph}
		\framesubtitle{Some further definitions}
		\begin{block}{Definition}
			\begin{description}
				\item[Path] A path is a trail in which all vertices (except possibly the first and last) are distinct.
				\item[Hamiltonian] A Hamiltonian path is a path that visits each vertex exactly once.
				\item[Ham. cycle] A cycle that visits each vertex exactly once (except for the start- and end-vertex)
			\end{description}
		\end{block}		
	\end{frame}
	
	\section{Approach}
	\begin{frame}
		\frametitle{Approach}
		\tableofcontents[
		sections={2},
		subsectionstyle=show/show/hide
		]
	\end{frame}
	
	\subsection{Graph notation}
	\begin{frame}
		\frametitle{Graph notation in Prolog}
		
		\begin{block}{Undirected graph}
		\end{block}
		\begin{block}{Directed graph}
		\end{block}
	\end{frame}
	
	\subsection{Finding paths in directed graphs}
	\begin{frame}
		\frametitle{Path finding}
		
	\end{frame}
	
	\subsection{Finding Hamiltonian paths}
	\begin{frame}
		\frametitle{Hamiltonian path}
		\begin{block}{Find Hamiltonian paths in directed graphs}
		\end{block}
	\end{frame}
	
	\subsection{Finding Hamiltonian cycles}
	\begin{frame}
		\frametitle{Hamiltonian cycle}
		\begin{block}{Find Hamiltonian cycles in directed graphs}
		\end{block}
	\end{frame}
	
	\subsection{Extension for undirected graphs}
	\begin{frame}
		\frametitle{Handling undirected graphs}
		\begin{block}{Naive approach}
			\begin{itemize}
				\item transformation into a directed graph by doubling edges
				\item inefficient due to larger search space
				\item prolog process ran out of memory (8 GB)
			\end{itemize}
		\end{block}
		\begin{block}{Efficient approach}
			\begin{itemize}
				\item check for edges in both directions 
			\end{itemize}
		\end{block}
	\end{frame}
	
	\section{Examples}
	\subsection{Directed graph}
	\begin{frame}
		\frametitle{Directed graph}
		\begin{columns}
			\begin{column}{0.3 \linewidth}
				\begin{figure}
				\end{figure}
			\end{column}
			\begin{column}{0.7 \linewidth}
				\begin{example}
				\end{example}
			\end{column}
		\end{columns}
	\end{frame}
	
	\subsection{Undirected graph}
	\begin{frame}
		\frametitle{Undirected graph}
		\begin{columns}
			\begin{column}{0.3 \linewidth}
				\begin{figure}
				\end{figure}
			\end{column}
			\begin{column}{0.7 \linewidth}
				\begin{example}
				\end{example}
			\end{column}
		\end{columns}
	\end{frame}
	
	\section{Questions}
	\begin{frame}
		\frametitle{Questions?}
	\end{frame}	
\end{document}