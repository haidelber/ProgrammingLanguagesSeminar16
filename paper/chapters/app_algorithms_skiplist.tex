\section{Algorithms for the Concurrent Skiplist}
\label{App-Algorithms-ConcurrentSkiplist}

In this section, we present the algorithms for the concurrent skiplist described in Sec.~\ref{Sec-Design-ConcurrentSkiplist}. The skiplist contains a Single-Writer-Multi-Readers lock with writer preference, called simply \texttt{lock}. In terms of notation, \texttt{lock.acquireR()} acquires the lock for reads, and \texttt{lock.acquireW()} acquires the lock for writes. The \texttt{SL::removeSeq()} skiplist procedure is described in Alg.~\ref{Alg-RemoveSeq}.

\begin{algorithm}[!htb]
\caption{SL::removeSeq()}
\label{Alg-RemoveSeq}
\begin{algorithmic}[1]
\If{minValue $=$ \maxInt}
	\State \Return \maxInt
\EndIf
\If{currSeq $=$ \nullvalue}
	\State moveHead()
\EndIf
\State key \attr currSeq.key
\State currSeq.counter \attr currSeq.counter - 1
\If{currSeq.counter $= 0$}
	\While{currSeq $\ne$ lastSeq}
		\State currSeq \attr currSeq.next[0]
		\If{currSeq.counter $> 0$}
			\State minValue = currSeq.key
			\State \Return key
		\EndIf
	\EndWhile
	\State moveHead()
\EndIf
\State \Return key
\end{algorithmic}
\end{algorithm}

The variable \texttt{lock.timestmap} contains the timestamp associated with the lock (and hence with the head-moving operations). Algorithm~\ref{Alg-CleanFind} returns a pair of elements $(b,r)$: $b$ is a bucket found using the skiplist \texttt{SL::find()} operation, and $r$ is a boolean defined as follows. If a head-moving operation happened anywhere between Lines~\ref{algFLT:checkstamp1} and~\ref{algFLT:checkstamp2}, the timestamp moved and $r$ will be false.

\begin{algorithm}[!htb]
\caption{cleanFind($v$, preds, succs)}
\label{Alg-CleanFind}
\begin{algorithmic}[1]
\State $t$ \attr lock.timestamp \label{algFLT:checkstamp1}
\State $b$ \attr find(headPar, $v$, preds, succs)
\State lock.acquireR()
\If{$t$ $<$ lock.timestamp} \label{algFLT:checkstamp2}
	\State lock.release()
	\State \Return $(\nullvalue, \mathbf{false})$
\EndIf
\State \Return $(b, \mathbf{true})$
\end{algorithmic}
\end{algorithm}

The \texttt{SL::addPar()} skiplist procedure is described in Alg.~\ref{Alg-AddPar}. It uses the clean find protocol above. It performs a clean find, followed by mutable operations (either increasing a counter or inserting a bucket), executed with \texttt{lock} acquired for reading.

\begin{algorithm}[!htb]
\caption{SL::addPar($v$)}
\label{Alg-AddPar}
\begin{algorithmic}[1]
\If{$v$ $\le$ lastSeq.key}
	\State \Return \textbf{false}
\EndIf
\State $(b,r)$ \attr cleanFind($v$, preds, succs) \label{algAddPar:find}
\If{$r$ = \textbf{false}}
	\State \textbf{restart} at line~\ref{algAddPar:find}
\EndIf
\If{$b \ne \nullvalue$} \label{algAddPar:incblock}
	\State Atomically increment $b$.counter
	\State lock.release()
	\State \Return \textbf{true}
\EndIf
\State $b$ \attr newNode($v$)
\For{$i$: 1 $\rightarrow$ $b$.topLevel}
	\State $b$.next[i] \attr succs[i]
\EndFor
\If{\textbf{not} CAS(preds[0].next[0]: succs[0] $\rightarrow$ $b$)}
	\State lock.release()
	\State \textbf{restart} at line~\ref{algAddPar:find}
\EndIf
\Repeat
	\State $m$ \attr minValue
\Until{$m \le v$ \textbf{or} CAS(minValue: $m \rightarrow v$)}
\For{$i$: 1 $\rightarrow$ $b$.topLevel}
	\State $b$.next[i] \attr succs[i]
	\If{CAS(preds[i].next[i]: succs[i] $\rightarrow$ $b$)}
		\State \Continue
	\EndIf
	\State lock.release()
	\Repeat
		\State $(b,r)$ \attr cleanFind($v$, preds, succs)
	\Until{$r = \mathbf{true}$}
	\If{$b = \nullvalue$}
		\State lock.release()
		\State \Return \textbf{true}
	\EndIf
\EndFor
\State \Return \textbf{true}
\end{algorithmic}
\end{algorithm}

The \texttt{SL::moveHead()} skiplist procedure is described in Alg.~\ref{Alg-MoveHead}. Line~\ref{algMH:copyHead} creates the sequential part starting from where the parallel part used to be, and the operations starting at Line~\ref{algMH:unlink} separate the skiplist in two parts. Note how \texttt{SL::find()} is used to locate the pointers that will change in order to separate the skiplist.

\begin{algorithm}[!htb]
\caption{SL::moveHead()}
\label{Alg-MoveHead}
\begin{algorithmic}[1]
\State $n$ is determined dynamically (see text)
\State lock.acquireW()
\State currSeq \attr \nullvalue

\State pred \attr headPar
\State curr \attr headPar.next[0]
\State $i = 0$
\While{$i < n$ \textbf{and} curr $\ne$ tail} \label{algMH:consume}
	\State i \attr i + curr.counter
	\If{currSeq = \nullvalue}
		\State currSeq \attr curr; minValue \attr curr.key
	\EndIf
	\State pred \attr curr; curr \attr curr.next[0]
\EndWhile

\If{$i = 0$}
	\For{$i: \maxLvl \rightarrow 0$}
		\State headPar[i], headSeq[i] \attr tail
	\EndFor

	\State lastSeq \attr headPar, minValue \attr \maxLvl
	\State lock.release()
	\State \Return \textbf{false}
\EndIf

\State lastSeq \attr pred

\For{$i: \maxLvl \rightarrow 0$} \label{algMH:copyHead}
	\State headSeq[i] \attr headPar[i]
\EndFor

\State find(headSeq, lastSeq + 1, preds, succs) \label{algMH:unlink}

\For{$i: \maxLvl \rightarrow 0$}
	\State preds[i].next[i] \attr tail
	\State headPar.next[i] \attr succs[i]
\EndFor

\State lock.release()
\State \Return \textbf{true}
\end{algorithmic}
\end{algorithm}

Finally, the \texttt{SL::chopHead()} skiplist procedure is described in Alg.~\ref{Alg-ChopHead}. Note that all the \texttt{SL::find()} operations are executed outside the critical section. These operations identify the pointers that will change in order to relink the skiplist.

\begin{algorithm}[!htb]
\caption{SL::chopHead()}
\label{Alg-ChopHead}
\begin{algorithmic}[1]
\If{currSeq = \nullvalue}
	\State \Return \textbf{false}
\EndIf

\State find(headSeq, lastSeq.key + 1, preds, \nullvalue)
\State find(headSeq, currSeq.key, \nullvalue, succs)

\State lock.acquireW()

\For{$i: \maxLvl \rightarrow 0$}
	preds[i].next[i] \attr headPar.next[i]
\EndFor

\State lastSeq \attr headPar, currSeq \attr \nullvalue

\For{$i: \maxLvl \rightarrow 0$}
	\State headPar.next[i] \attr succs[i] \textbf{if} succs[i] $\ne$ tail
\EndFor

\State lock.release()

\State \Return \textbf{true}
\end{algorithmic}
\end{algorithm}
