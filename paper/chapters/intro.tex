% !TEX root = ../paper.tex

\section{Introduction}

\unsure{Give context to the paper.. long intro.. skip some of the main part}

A priority queue is a data structure for storing elements with associated keys. The keys represent priorities. Priority queues can be implemented in two forms: max-priority queues and min-priority queues. The latter one is the form that is considered in this paper.
Typically a min-priority queue exports two operations: \texttt{add()}, for inserting an item into the priority queue, and \texttt{removeMin()} for removing the element with the minimum priority. \citeauthor{cormen_introduction_2009} describe other typical operations of a priority queue like: \texttt{minimum()} for retrieving without removing the minimum-priority-element, and \texttt{decreaseKey()} for decreasing the key of the given element to a given value.
(Parallel) priority queues are often used for resource management in schedulers, for event simulations, or as data structures in graph-algorithms (e.g. Dijkstra's shortest path algorithm, or Prim's minimum spanning tree algorithm) \cite{cormen_introduction_2009}.

\subsection{Prior Work}

The first parallel priority queues implementations were based on heaps and linked lists as it can be seen in the study by \citeauthor{ronngren_comparative_1997} \cite{ronngren_comparative_1997}, parallel priority queue algorithms based on skiplists were proposed later by \citeauthor{shavit_scalable_1999} \cite{shavit_scalable_1999}.

\citeauthor{lotan_skiplist-based_2000} \cite{lotan_skiplist-based_2000} and \citeauthor{shavit_scalable_1999} \cite{shavit_scalable_1999}
\improve{Write about skip list based prior work }

\citeauthor{sundell_fast_2003} \cite{sundell_fast_2003}
\improve{Write about lock free prior work}

\citeauthor{hendler_flat_2010} \cite{hendler_flat_2010}
\improve{Write about flat combining to reduce contention}

\citeauthor{hendler_scalable_2004} \cite{hendler_scalable_2004}
\improve{Write about elimination}

\subsection{Contributions}

\subsection{Basics}

\change{Explain skiplist}

\change{Explain hardware transactional memory}