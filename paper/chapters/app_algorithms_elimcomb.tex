\section{Algorithms for Elimination and Combining}
\label{App-Algorithms-EliminationCombining}

In this section, we present the algorithms for the elimination and combining strategies for our priority queue, described in Section~\ref{Sec-Design-EliminationCombining}. The priority queue removal is shown in Alg.~\ref{Alg-removeMin}.

% \begin{algorithm}[htb]
% \caption{PQ::removeMin()}
% \label{Alg-removeMin}
% \begin{algorithmic}[1]
% \While{$\mathbf{true}$}
% 	\State pos \attr $(id + 1) \% $ ELIM\_SIZE; (value, stamp) \attr elim[pos]
% 	\If {IsValue(value) \textbf{and} (stamp $> 0$) \textbf{and} (value $\le$ skiplist.minValue))}
% 	    \If{CAS(elim[pos], (value, stamp), (TAKEN, 0))}
% 		\State \Return value
% 	    \EndIf
% 	\EndIf
% 	\If {value = EMPTY}
% 	    \If{CAS(elim[pos], (value, stamp), (REMREQ, uniqueStamp()))}
% 		\Repeat
% 		  \State (value, stamp) \attr elim[pos]
% 		\Until{value $\ne$ REMREQ \textbf{and} value $\ne$ INPROG}
% 		\State elim[pos] \attr (EMPTY, 0); \Return value
% 	    \EndIf    
% 	\EndIf
% 	\State inc(pos)
% \EndWhile
% \end{algorithmic}
% \end{algorithm}






% \begin{algorithm}[htb]
% \caption{PQ::add(inValue)}
% \label{Alg-add}
% \begin{algorithmic}[1]
% \If{skiplist.addPar(inValue)}
%     \State \Return $\mathbf{true}$
% \EndIf
% 
% \While{$\mathbf{true}$}
% 	\State (value, stamp) \attr elim[pos]
% 	\If {value = REMREQ \textbf{and} (inValue $\le$ skiplist.minValue))}
% 	    \If{CAS(elim[pos], (value, stamp), (inValue, 0))}
% 		\State \Return $\mathbf{true}$
% 	    \EndIf
% 	\EndIf
% 	\If {value = EMPTY}
% 	    \If{CAS(elim[pos], (value, stamp), (inValue, uniqueStamp()))}
% 		\Repeat
% 		  \State (value, stamp) \attr elim[pos]
% 		\Until{value = TAKEN}
% 		\State elim[pos] \attr (EMPTY, 0); \Return $\mathbf{true}$
% 	    \EndIf    
% 	\EndIf
% 	\State inc(pos)
% \EndWhile
% \end{algorithmic}
% \end{algorithm}

The priority queue insertion algorithm is shown in Alg.~\ref{Alg-add}. If the value being inserted is not suitable for the parallel part (\texttt{PQ::addPar()} returns false), the request is posted in the elimination array, until eliminated with a suitable \texttt{PQ::removeMin()} or consumed by the server thread. Details are discussed in Sec.~\ref{Sec-Design-EliminationCombining}. 

\begin{algorithm}[!htb]
\caption{PQ::add(inValue)}
\label{Alg-add}
\begin{algorithmic}[1]
\If{inValue $\le$ skiplist.minValue}
  \State rep \attr MAX\_ELIM\_MIN
\Else 
  \If{skiplist.addPar(inValue)}
    \State \Return $\mathbf{true}$
  \EndIf
  \State rep = MAX\_ELIM
\EndIf

\While{rep $> 0$}
	\State pos \attr $(id + 1) \% $ ELIM\_SIZE; (value, stamp) \attr elim[pos]
	\If {value = REMREQ \textbf{and} (inValue $\le$ skiplist.minValue))}
	    \If{CAS(elim[pos], (value, stamp), (inValue, 0))}
		\State \Return $\mathbf{true}$
	    \EndIf
	\EndIf
	\State rep \attr rep $- 1$; inc(pos)
\EndWhile

\If{skiplist.addPar(inValue)}
    \State \Return $\mathbf{true}$
\EndIf

\While{$\mathbf{true}$}
	\State (value, stamp) \attr elim[pos]
	\If {value = REMREQ \textbf{and} (inValue $\le$ skiplist.minValue))}
	    \If{CAS(elim[pos], (value, stamp), (inValue, 0))}
		\State \Return $\mathbf{true}$
	    \EndIf
	\EndIf
	\If {value = EMPTY}
	    \If{CAS(elim[pos], (value, stamp), (inValue, uniqueStamp()))}
		\Repeat
		  \State (value, stamp) \attr elim[pos]
		\Until{value = TAKEN}
		\State elim[pos] \attr (EMPTY, 0); \Return $\mathbf{true}$
	    \EndIf    
	\EndIf
	\State inc(pos)
\EndWhile
\end{algorithmic}
\end{algorithm}

% The server thread executes Alg.~\ref{Alg-Execute}, shown below. The algorithm implements the combining paradigm described in Sec.~\ref{Sec-Design-EliminationCombining}.
% \begin{algorithm}[htb]
% \caption{Server::execute()}
% \label{Alg-Execute}
% \begin{algorithmic}[1]
% \While{$\mathbf{true}$}
%   \For{$i$: 1 $\rightarrow$ ELIM\_SIZE}
% 	\State (value, stamp) \attr elim[i]
% 	\If{value = REMREQ}
% 	    \If{CAS(elim[i], (value, stamp), (INPROG, 0))}
% 		\State min \attr skiplist.removeSeq(); elim[i] \attr (min, 0)
% 	    \EndIf
% 	\EndIf
% 	\If {IsValue(value) \textbf{and} (stamp $> 0$)}
% 	    \If{CAS(elim[i], (value, stamp), (INPROG, 0))}
% 		\State skiplist.addSeq(value); elim[i] \attr (TAKEN, 0)
% 	    \EndIf    
% 	\EndIf
%   \EndFor
% \EndWhile
% \end{algorithmic}
% \end{algorithm}
